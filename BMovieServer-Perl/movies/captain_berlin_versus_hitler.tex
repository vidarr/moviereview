\documentclass [11pt]{article}
\usepackage[utf8]{inputenc}
\usepackage[ngerman]{babel}
\usepackage {a4wide}
\usepackage{bmovie}
\begin{document}
\bmUnterhaltungswert[]{4}
\bmPornofaktor[]{1}
\bmGewaltdarstellung[]{1}
\bmGewaltverherrlichung[]{1}
\bmNiveau[]{4}
\bmSexismus[]{2}
\bmRealismus[]{1}
\bmTechnisch{Schatten des Mikrofons im Bild.}
\bmWissenschaft{Das Stück stellt eine surrealistische Welt dar. Wissenschaftliche Kriterien sind hier nur eingeschränkt verwendbar; 1973 hatte der Reichstag noch keine gläserne Kuppel.}
\bmInhalt{Das Stück stellt eine surrealistische Welt dar. Inhaltlich-logische Kriterien sind hier nur eingeschränkt verwendbar.}
\bmBild{Der Geist Adolf Hitlers lebt weiter. Die Bedrohung durch den Faschismus besteht fort.}
\bmHandlung{Die ehemalige Leibärztin Hitlers hat sein Gehirn aufbewahrt und am Leben erhalten. 1973 versucht sie dieses zunächst in einen Leichnam, anschließend in einen Roboter einzusetzen um ihren Führer so wieder zum Leben zu erwecken. Doch Hitlers alter Erzfeind Captain Berlin, ein während des 3. Reichs vom Untergrund erschaffener Superheld – stellt a sich der Wissenschaftlerin in den Weg.}
\bmBemerkungen{Der Film ist ein abgefilmtes, nachbearbeitetes Theaterstück; Graf Dracula liegt im kommunistischen Brandenburg begraben und zeigt deutliche marxistische Anwandlungen; ``Aus Leichenteilen gefallener, deutscher Landser zusammengeflickter Zuchtbulle”; Hitlerrobo aus gehärtetem Kruppstahl; Fön als ''Ultrarevolver”.}
\bmZitate{
``Ich habe es geahnt - Adolf Hitler lebt!''
\\
``Der Vampir ist aus dem Ostblock. Er wird sich seine Wohnung also in der Ostzone suchen.''
}
\bmMakeBMovieBogen{Captain berlin versus Hitler}{Deutschland}{Captain Berlin versus Hitler}{2008}{unbekannt}{Drama}
\end{document}
