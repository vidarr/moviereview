\documentclass [11pt]{article}
\usepackage[utf8]{inputenc}
\usepackage[ngerman]{babel}
\usepackage {a4wide}
\usepackage{bmovie}
\begin{document}
\bmUnterhaltungswert[]{2}
\bmPornofaktor[]{2}
\bmGewaltdarstellung[]{3}
\bmGewaltverherrlichung[]{3}
\bmNiveau[]{1}
\bmSexismus[]{2}
\bmProfessionalitaet[]{2}
\bmRealismus[]{2}
\bmTechnisch{Soundeffekte klingen absolut künstlich; Aufnahmegeräusche; Kamera schwankt; schauspielerische Leistung unterirdisch.}
\bmWissenschaft{Nach Stich Messer blank.}
\bmInhalt{Feuer brennt einfach unbeaufsichtigt, Mann kommt an und setzt sich ohne sich zu wundern.}
\bmBild{Nixon ist das Böse und Death metal sein Werkzeug.}
\bmHandlung{Brad, ein eingefleischter Death Metal Fan, gewinnt bei einem Radiosender ein Tonband seiner Lieblingsband ``Living Corpse''. Leider wird schnell deutlich, dass jeder, der das Band hört, zum Zombie wird. Der Bandleader offenbart sich schließlich als  der Leibhaftige, der mittels des Bandes eine große willenlose Anhängerschar gewinnen will, um mit Ihnen seine Herrschaft über die Welt zu errichten. Schließlich gelingt es einigen Überlebenden, den Bann über die Zombies mittels geeigneter ``Gegenmusik'' zu brechen.}
\bmBemerkungen{``Play this movie loud!''; extreme Längen; dünnes Drehbuch/ sinnlose Dialoge; Nixon tritt als Killer auf, der durch den Film führt; Prügelei klingt nach Kissenschlacht; erwacht nach Feier mit toter Katze in der Hand , wirft diese aber als wäre es das Normalste der Welt einfach in Zimmerecke; Zombie geht an Telefon; Bestimmte Art Musik heilt Zombies; Sinn einiger Szenen liegt im Dunkeln/ist nur schwer erfassbar; Zombies sind intelligent, nutzen Werk- und Schlachtzeug, versuchen, mittels Benzin Haus in Brand zu setzen, machen Witze.}
\bmZitate{
``Stay away from me or I play country music till your ears bleed!''
\\
``How long is the song?'' - ``About 13 minutes'' - ``Then we got time and zombies to kill!''
}
\bmMakeBMovieBogen{Death Metal Zombies}{VSA}{Death Metal Zombies}{1995}{unbekannt}{Horror}
\end{document}
