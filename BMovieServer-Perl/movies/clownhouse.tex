\documentclass [11pt]{article}
\usepackage[utf8]{inputenc}
\usepackage[ngerman]{babel}
\usepackage {a4wide}
\usepackage{bmovie}
\begin{document}
\bmUnterhaltungswert[]{2}
\bmPornofaktor[]{1}
\bmGewaltdarstellung[]{2}
\bmGewaltverherrlichung[]{2}
\bmNiveau[]{1}
\bmSexismus[]{1}
\bmProfessionalitaet[]{4}
\bmRealismus[]{4}
\bmTechnisch{Clown, der durchs Fenster fliegt, sieht man an, dass es eine Puppe ist.}
\bmWissenschaft{Während der Zirkusvorstellung hört man vor dem Zelt nichts von drinnen.}
\bmInhalt{anche Verhaltensweisen der Figuren erscheinen nicht wirklich logisch.}
\bmBild{Man muss sich seinen Ängsten stellen.}
\bmHandlung{Drei Ausbrecher aus der Psychiatrie morden in Clownskostümen. Nur ein ängstlicher, kleiner Junge nimmt sie zunächst wahr, doch niemand glaubt ihm.}
\bmBemerkungen{Puppe wird zu Halloween im Garten aufgehängt; bei Nachtszenen wird viel zu viel Scheinwerfen verwendet.}
\bmZitate{
Hast du nicht eine Kleinigkeit vergessen? Zum Beispiel wer dich verprügeln kann?
}
\bmMakeBMovieBogen{Clownhouse}{VSA}{Clownhouse}{1989}{18}{Slasher}
\end{document}
