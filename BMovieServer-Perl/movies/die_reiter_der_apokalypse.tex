\documentclass [11pt]{article}
\usepackage[utf8]{inputenc}
\usepackage[ngerman]{babel}
\usepackage {a4wide}
\usepackage{bmovie}
\begin{document}
\bmUnterhaltungswert[Breite Fächerung]{4}
\bmPornofaktor[]{1}
\bmGewaltdarstellung[]{3}
\bmGewaltverherrlichung[]{2}
\bmNiveau[]{3}
\bmSexismus[]{1}
\bmProfessionalitaet[]{5}
\bmTechnisch{Dem Publikum ist nichts aufgefallen.}
\bmWissenschaft{Dem Publikum ist nichts aufgefallen.}
\bmInhalt{Vater kauft Sohn für sicherlich mehrere Monatslöhne Pony als Geschenk, gibt als Grund für Bleiben auf der Ranch jedoch Arbeitslohn  als Knecht an, Geld, ``dass wir gebrauchen können''.}
\bmBild{Bibelpropaganda? Film wirkt teils missionarisch.}
\bmHandlung{Den kleinen Samuel quälen düstere Visionen von entstellten Toten. Psychologen führen diese auch darauf zurück, dass sich seine Eltern getrennt haben, können ihn allerdings nicht davon befreien. Als er mit seinem Vater zusammen nach einem Unfall von einem Bauern gerettet und auf dessen Hof gebracht wird, entschließt sich sein Vater, sich dort als Knecht zu verdingen. Der auf den ersten Blick recht sympathische Bauer erscheint Samuel allerdings zunehmend bedrohlich. Jener erringt mehr und mehr Einfluss über den Vater, bindet ihn durch die anscheinend gut bezahlte Arbeitsstelle und einigen leichten Mädchen immer stärker an den Hof.  Derweil verstärken sich Samuels Visionen, im Badezimmer glaubt er die Decke sich blutrot färben zu sehen, die Glühbirne an derselben scheint voll Blut zu laufen. Immer öfter sieht er wandelnde Tote mit vernähten Mündern. Auch in der Schule findet er keinen Anschluss an die anderen Kinder, wird von einem sogar regelrecht gehetzt. Einzig seine Religionslehrerin scheint ihm Verständnis entgegenzubringen. Sie offenbahrt ihm, dass einer Theorie zufolge die Apokalypse zum Vorteil Satans ``umgekehrt'' werden kann, sollte -nach Adam und Eva- ein weiteres Mal vom Baum des Lebens gegessen werden. Wie Samuel scheint auch dessen Therapeutin zu spüren, dass Böses vom Bauern ausgeht. Ihr Besuch auf dem Hof endet allerdings damit, dass Samuel im Wahn glaubt, den Bauern sie erschlagen zu sehen. In der Scheune glaubt er kurze Zeit später auf einen regelrechten Seziersaal, in welchem der Bauer einer Leiche den Mund vernäht zu stoßen. Wenig später sieht er den Bauern mit dem Anhänger, den er der Psychologin nach ihrem Tod abgenommen hat. Es gelingt Samuel, diesen an sich zu bringen. Als auch diese Tote ihm mit vernähtem Mund erscheint, gibt er ihr den Anhänger, woraufhin sich die Naht löst und sie ihn über das Wesen dieser wandelnden Toten aufklärt: Der Bauer sammelt Seelen, und vernäht ihnen den Mund, auf dass sie keine höheren Mächte zu Hilfe rufen können. Nichtsdestotrotz weigert sich sein Vater,den lukrativen Arbeitsplatz aufzugeben, auch weil er die Beobachtungen und Visionen seines Sohnes für bloße Hirngespinste ansieht. Schließlich prügelt er diesen vom Bauer angestachelt gar mit seinem Gürtel, um ihm seine Fantastereien auszutreiben. Samuel erkennt schließlich die Rolle, welche der Bauer ihm und seinem Vater zugedacht hat: Einer von ihnen soll vom Baum des Lebens essen und damit die Apokalypse umkehren! Immer öfter kommt es dazu, dass der Bauer Unheil schafft, Samuel aber nur tatenlos zusehen kann, da stets ein Wolf auftaucht, der ihn in Schach hält. Schließlich sieht er den Bauern mit seinem Vater im Garten bei einem Birnenbaum. Ihm ist inzwischen klar geworden, dass dies der Baum des Lebens ist - der Bauer ist dabei, seinen Vater zum Pflücken einer Frucht zu bewegen - und wieder erscheint der Wolf. Diesmal jedoch tritt er ihm entschlossen entgegen, woraufhin dieser mit jaulend das Weite sucht. Im Garten angekommen, preschen vier grauenhafte Reiter aus dem Stall - die Reiter der Apokalypse. Schon legt sein Vater Hand an eine der Früchte des Baumes...}
\bmBemerkungen{Der Film scheint sich lose an die durch die Offenbahrung vorgegebenen Handlungsrahmen zu orientieren: Versuchung der Welt durch den Teufel (z.B. in Form leichter Mädchen) sowie Kampf und Vernichtung desselben und letztendlich das Heraufziehen eines neuen Paradieses.\\ Dabei setzt der Film viele Details betont in Szene, wodurch er dem geneigten Betrachter durchaus weitgehende Interpretationsmöglichkeiten gibt. Vieles könnte als Verweis auf ``klassische'' Literatur gewertet werden. Die Glühbirne, die sich im Badezimmer mit Blut füllt erinnert an die Passage ``...und der Mond wurde wie Blut'' in der Offenbarung nach dem Öffnen des sechsten Siegels. Eine Bronzefigur mit Lorbeerkranz, der eigentlich keine Bedeutung zukommt, wird mehrmals in Großaufnahme gezeigt. Handelt es sich hierbei um ein Abbild Dantes? Ebenso wird einem Globus, den ``der Teufel'' in Drehung versetzt, eigentlich unnötig viel Aufmerksamkeit gewidmet. Warum ist es ausgerechnet ein Wolf, welcher den Jungen wieder und wieder daran hindert, in entscheidenden Momenten zu handeln? Warum bleibt es bei der Bedrohung, ja warum flieht dieser mit im wahrsten Sinne des Wortes mit ``eingezogenem Schwanz'' als der Junge sich nicht mehr von ihm einschüchtern lässt? Leider scheint der Film es beim mehr oder weniger losem Aneinanderreihen solcher symbolträchtiger Bilder bewenden zu lassen und zitiert teils (zu) frei: Warum fällen die Reiter der Apokalypse den Teufel? Und was hat es mit der vom Satan angestrebten ``Umkehr der Apokalypse'' auf sich? Vieles wirkt letztendlich ohne tieferen Sinn lediglich des Effektes wegen eingebaut zu sein; Merkwürdige Wahl des Lehrstoffes im Unterricht: Nur erstes (Schöpfungsgeschichte) und letztes Buch (Offenbahrung des Johannes) wird behandelt.}
\bmZitate{
``Wohnen Sie hier schon ein Leben lang?'' - ``Noch nicht''
\\
``Ein Mann für eine Frau? - Ein Witz für uns alle!''
\\
``Die Apokalypse ist Gottes Art dir zu zeigen, dass er genug hat.''
}
\bmMakeBMovieBogen{Die Reiter der Apokalypse}{VSA}{The Garden}{2006}{16}{Horror / Drama}
\end{document}
